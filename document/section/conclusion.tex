\section*{Conclusioni}
L'architettura di NELL utilizza 4 componenti per imparare ad estrarre conoscenza utilizzando modalità differenti per massimizzare l'accuratezza dei risultati.
Dopo soli 67 giorni, la base di dati è arrivata a contare oltre 242.000 fatti, con una precisione complessiva di circa 74\%.\newline
Risultati simili non sarebbero stati possibili utilizzando approcci tradizionali. L'esito positivo di questo progetto non sarebbe stato possibile senza algoritmi di Machine Learning combinati tra loro attraverso un'architettura in grado di massimizzare la precisione dei risultati complessivi.\newline\newline
\noindent Il server su cui NELL computava è stato spento e risulta irraggiungibile dall'estate del 2018. Non è più possibile fornire feedback di miglioramento, ma resta ancora aperta la possibilità di consultare la knowledge base attraverso la funzionalità denominata \textit{Knowledge Base Browser} all'indirizzo \url{http://rtw.ml.cmu.edu/rtw/kbbrowser/}.\newline
\newline
Il progetto nella sua interezza non risulta eccessivamente datato, ma sono sicuramente possibili alcuni miglioramenti. La libreria utilizzata per memorizzare la KB risulta ormai deprecata. La migrazione dell'enorme quantità di dati raccolti da NELL ad un nuovo sistema di memorizzazione non è un'operazione semplice, e sarebbe necessaria un processo di analisi per verificare se è conveniente.\newline
Le credenze raccolte da NELL sono innumerevoli, ma non tutte sono corrette. Questo problema è possibile da mitigare grazie alla supervisione umana, che seppur estremamente costosa e lenta è ancora oggi uno dei metodi più affidabili per identificare gli errori e permettere agli algoritmi di migliorare.
\newpage