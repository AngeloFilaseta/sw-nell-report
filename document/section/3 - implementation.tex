\section{Implementazione}
\label{impl}
\subsection{Subsystem Components}
I quattro componenti utilizzati da NELL hanno tutti come obiettivo comune l'identificazione di nuovi fatti candidati. Ogni componente però utilizza tecniche differenti. In questo modo è possibile aumentare l'accuratezza dei risultati. I Subsystem Components sono:
\begin{itemize}
	\item \textbf{Coupled Pattern Learner (CPL)}: Un componente che si occupa di estrazione del testo, utilizzando pattern testuali come "maggiore di X" o "X gioca in Y" per estrarre istanze di categorie e relazioni. Utilizza un algoritmo di Machine Learning che accoppia il learning semi-supervisionato di categorie e relazioni. Approcci più tradizionali non riescono a classificare in modo efficace i dati quando la differenza di cardinalità tra dati etichettati e non etichettati è elevata.
	CPL utilizza statistiche di occorrenza tra i sostantivi e alcuni pattern (grazie al PoS tagging) per imparare pattern di estrazione per ogni predicato di interesse. Vengono poi utilizzati gli stessi pattern per trovare nuovi predicati.
	Anche l'utilizzo di relazioni tra i predicati vengono utilizzate per una migliore valutazione complessiva.
	Le probabilità di ogni fatto candidato estratto da questo componente vengono calcolate utilizzando la formula $1 - 0.5^{c}$ dove $c$ è il numero di pattern che hanno estratto il candidato.
	\begin{info}[PoS tagging]
		Il Part of Speech Tagging, anche chiamato tagging grammaticale, è il processo di marcatura ed etichettatura di una parola in un testo come corrispondente a una parte particolare del discorso, basata sia sulla sua definizione che sul suo contesto\cite{POStagsa85:online}.
	\end{info}
\begin{code}
\captionof{listing}{Descrizione dell'algoritmo CPL \cite{Coupledp1:online}}
\begin{minted}{java}
// Input: An ontology O, and a text corpus C 
// Output: Trusted instances/patterns for each predicate
while(true) {
	for(Predicate p : Ontology o) {
		/*
		1) EXTRACT candidate instances/contextual patterns 
		using recently promoted patterns/instances;
		2) FILTER candidates that violate coupling;
		3) RANK candidate instances/patterns;
		4) PROMOTE top candidates;
		*/
	}
}
\end{minted}
\end{code}
\end{itemize}

\newpage