\section{Criticità}

\subsection{Criticità algoritmiche}
\textbf{Problema}\newline
L'approccio utilizzato per il training del sistema soffre particolarmente in situazioni in cui gli errori di labeling cominciano ad accumularsi.\newline\newline
\textbf{Strategia di mitigazione}\newline
Per mitigare questo problema, ogni giorno viene eseguita un'operazione di revisione di circa 10-15 minuti da parte di utenti umani. A fini dimostrativi si è comunque cercato di limitare il più possibile l'intervento umano\cite{TowardAnArchitecture:online}.

\subsection{Criticità nei risultati}
\textbf{Problema}\newline
Come già ribadito più volte, NELL non è perfetto. La knowledgebase cresce molto velocemente e la possibilità che l'algoritmo commetta errori di labeling è bassa ma non 0. Vengono di seguito riportate alcune voci trovate all'interno della knowledge base che sono state categorizzate come credenze ma che in realtà contengono informazioni errate.

\centeredImage{img/false\_result.png}{Credenze false\cite{ReadtheWeb:online}}{1}
\noindent In figura, il primo statement indica in forma testuale la credenza, il numero subito a destra è l'iterazione in cui NELL ha trovato il fatto, seguito subito dalla data. Il valore rimanente esprime con quanta probabilità l'informazione è corretta. I due pulsanti rimanenti servono a fornire un feedback per aiutare l'algoritmo a migliorare.\newline\newline
\textbf{Strategia di mitigazione}\newline
La knowledge base è consultabile anche attraverso browser all'indirizzo \url{http://rtw.ml.cmu.edu/rtw/kbbrowser/} ed è possibile per gli utenti fornire feedback su ogni credenza. Gli algoritmi alla prossima iterazione cercheranno di migliorarsi, utilizzando i valori dei feedback come ulteriori pesi per il training.
\newpage