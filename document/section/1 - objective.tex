\section{Obiettivi}

\subsection{Scopo del progetto}
La maggior parte dei sistemi di Machine Learning acquisiscono informazioni riguardanti un singolo dominio specifico. La maggior parte dei modelli si basano su un singolo data set strutturato.

\noindent Il team del progetto "Read The Web" propone un paradigma di apprendimento "infinito" denominato NELL (\textit{Never-Ending Language Learning}), il  cui scopo è memorizzare qualsiasi informazione, indipendentemente dal tipo o dal dominio. Scopo del progetto è analizzare le pagine web presenti su internet, ottenendo una knowledge base condivisa, strutturata, consultabile ed interrogabile.
\subsection{Storia del progetto}
NELL ha cominciato la sua computazione nel Gennaio 2010, ma solo dopo 6 mesi è diventato completamente indipendente.

\noindent Durante la fase in cui NELL ha avuto necessità di supervisione umana, sono state estratte circa 350.000 istanze tra categorie e relazioni. Già a quel punto, NELL era in grado di riconosce più di tre quarti di quelle categorie e relazioni con una precisione che si aggirava tra il 90\% e il 99\%. La parte rimanente dell'ontologia veniva riconosciuta molto più difficilmente, in un range di precisione che andava dal 25\% al 60\%. La precisione totale dell'intera knowledge base aveva raggiunto in poco tempo un'affidabilità piuttosto alta. Circa il 71\% delle informazioni erano corrette.

\noindent Per alcuni tipi di informazioni l'algoritmo risulta ancora oggi piuttosto instabile:
\begin{itemize}
    \item molte informazioni vengono categorizzate in modo inizialmente errato, risultando attendibili solo dopo un certo numero di iterazioni;
    \item alcune informazioni perdono sensibilmente accuratezza con l'avanzare del tempo.
\end{itemize}

\noindent A partire da Giugno 2010, la fase di manutenzione è cambiata, basandosi su controlli periodici settimanali.I principali task consistevano nel controllare ogni categoria scannerizzata per un almeno 5 minuti, un tempo più che sufficiente per capire se NELL fosse in grado di estrapolare correttamente informazioni per la categoria analizzata. Scopo delle review era segnalare quali fossero gli errori più significativi e gravi, al fine di permettere all'algoritmo di migliorare. I feedback da parte di utenti umani ancora oggi aiutano NELL a migliorare.

\centeredImage{img/nell-timeline.png}{Numero di asserzioni confermate da NELL nel tempo \cite{ReadtheWebOverview:online}}{0.6}

\noindent Nella sezione dedicata alla knowledge base sul sito vengono riportati tutti i fatti che hanno superato un valore di accuratezza di oltre 90\%.\newline
Un qualsiasi utente può fornire un feedback sulla correttezza degli statement, in modo da permettere a NELL di migliorare all'iterazione successiva.
Una volta raggiunta l'accuratezza dell'87\% su tutti i fatti presenti nella knowledge base, sono state aggiunte più categorie e relazioni \cite{ReadtheWebOverview:online}.
Ad oggi, NELL ha accumulato oltre 50 milioni di asserzioni, anche se solo poco meno di 3 milioni risultano accurate, NELL migliora col passare del tempo \cite{ReadtheWeb:online}.
\newpage