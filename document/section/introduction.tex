\section*{Introduzione}
Può un calcolatore imparare a leggere? Secondo il team di ricerca del progetto "Read the Web" la risposta è sì.\newline
NELL (\textit{Never-Ending Language Learning}) è un sistema intelligente che raccoglie informazioni dal web, estraendo dei "fatti" che vengono ritenuti significativi e corretti, ed inserendoli un una knowledge base condivisa. Il nome del progetto indica proprio un sistema (o un agente) che esegue 24 ore su 24 (per sempre) e che impara continuamente e in modo sempre più efficiente ed efficace.

\noindent Dal 2010 NELL è in esecuzione, e da allora sono stati accumulati oltre 50 milioni di "fatti candidati" a diversi livelli di attendibilità.\newline
Non tutti i fatti raccolti sono sicuri ed accurati, solo circa 3 milioni dei fatti raccolti risultano effettivamente delle "credenze".

\noindent Il sistema non è sicuramente perfetto, ma un punto di forza di NELL è proprio la sua capacità di imparare dagli errori commessi per risultare più accurato in futuro.\newline
Il progetto, denominato "Read the Web" nasce dalla ricerca condotta da un team dell'Università Carnegie Mellon, in Pennsylvania \cite{ReadtheWeb:online}.

\newpage