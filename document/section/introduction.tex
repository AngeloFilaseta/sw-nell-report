\section*{Introduzione}
Può un calcolatore imparare a leggere? Secondo il team di ricerca che ha creato NELL la risposta è sì.\newline
NELL (\textit{Never-Ending Language Learning}) è un sistema intelligente che raccoglie informazioni dal web, estraendo dei "fatti" che vengono ritenuti significativi e validi.
Dal 2010 NELL è in esecuzione, e da allora sono stati accumulati oltre 50 milioni "fatti candidati" a diversi livelli di attendibilità.\newline
Non tutti i fatti raccolti sono sicuri ed accurati, solo circa 3 milioni dei fatti raccolti risultano effettivamente validi.
Il sistema non è sicuramente perfetto, ma un punto di forza di NELL è proprio la sua capacità di imparare dagli errori commessi per essere più accurato in futuro.\newline
Il progetto, denominato "Read the Web" nasce dalla ricerca condotta da un team dell'Università Carnegie Mellon, in Pennsylvania \cite{ReadtheWeb:online}.
\newpage